\documentclass{dcbl/challenge}
%%% -- {
%%% --     "metastyle": "0.0.1",
%%% --     "meta": {
%%% --          "uuid": "92e5a3bc-4da7-4fde-89f8-5c8fe911c1e1",
%%% --          "title": "First Steps with VHDL",
%%% --          "language": "en",
%%% --          "authors": [
%%% --              {
%%% --                  "name": "Stephan Bökelmann",
%%% --                  "email": "sboekelmann@ep1.rub.de",
%%% --                  "affiliation": "AG Physik der Hadronen und Kerne"
%%% --              }
%%% --          ],
%%% --          "depends on": {},
%%% --          "recommends": [],
%%% --          },
%%% --          "tags": {
%%% --              "primary": ["Geschichte", "PDP-7"],
%%% --              "secondary": [],
%%% --              "category": ["Geschichte"],
%%% --         },
%%% --     }
%%% -- }
\setdoctitle{Geschichte der Rechenmaschinen}
\setdocauthor{Stephan Bökelmann}
\setdocemail{sboekelmann@ep1.rub.de}
\setdocinstitute{AG Physik der Hadronen und Kerne}


\begin{document}

Die Geschichte der Rechenmaschinen, motiviert unser heutiges Verständnis und die heutige Anwendung moderner Computer und der Digitalelektronik im Allgemeinen. 
Ohne ein Verständnis der Grundlagen dieser Maschinen, wird es auch schwierig, tiefer in die moderne Informatik einzusteigen.
Daher sollen die folgenden Aufgaben Ihnen die Möglichkeit geben genauer zu erforschen, was die Essenz eben dieser Maschinen ist, welche mehr und mehr Einzug in unser Leben erhalten.

\section*{Aufgaben}
\begin{aufgabe}
    Um die Mächtigkeit von Rechenmaschinen zu beschreiben bedienen wir uns hauptsächlich zweier Grundbegriffe:
    \begin{enumerate}
        \item FLOPS
        \item Hauptspeicher (RAM)
    \end{enumerate}
    Definieren Sie die beiden Begriffe und beschreiben Sie, warum genau diese Begriffe bedeutend für die Leistung von Rechenmaschinen sind.
\end{aufgabe}

\begin{aufgabe}
    Was ist Moore's Law und welche Aussage macht es über die Rechenleistung zukünftiger Maschinen?
\end{aufgabe}

\begin{aufgabe}
    Recherchieren Sie den \textit{DEC PDP-7} und fassen Sie schriftlich zusammen, welche Gemeinsamkeiten Sie zu modernen Rechenmaschinen sehen.
\end{aufgabe}

\section*{Anmerkungen}
\begin{enumerate}
    \item PDP-7 läuft mit UNIX-v0: \url{https://www.youtube.com/watch?v=pvaPaWyiuLA}
    \item Emulation einer UNIX-v7 Instanz auf einem PDP-11 Emulator: \url{https://www.youtube.com/watch?v=vERIL4JpFPU}
    \item Erklärung der von-Neumann-Architektur durch Prof. Brailsford der Uni Nottingham: \url{https://www.youtube.com/watch?v=Ml3-kVYLNr8}
\end{enumerate}

\end{document}
